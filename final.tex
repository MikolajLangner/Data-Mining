\documentclass{article}\usepackage[]{graphicx}\usepackage[]{color}
% maxwidth is the original width if it is less than linewidth
% otherwise use linewidth (to make sure the graphics do not exceed the margin)
\makeatletter
\def\maxwidth{ %
  \ifdim\Gin@nat@width>\linewidth
    \linewidth
  \else
    \Gin@nat@width
  \fi
}
\makeatother

\definecolor{fgcolor}{rgb}{0.345, 0.345, 0.345}
\newcommand{\hlnum}[1]{\textcolor[rgb]{0.686,0.059,0.569}{#1}}%
\newcommand{\hlstr}[1]{\textcolor[rgb]{0.192,0.494,0.8}{#1}}%
\newcommand{\hlcom}[1]{\textcolor[rgb]{0.678,0.584,0.686}{\textit{#1}}}%
\newcommand{\hlopt}[1]{\textcolor[rgb]{0,0,0}{#1}}%
\newcommand{\hlstd}[1]{\textcolor[rgb]{0.345,0.345,0.345}{#1}}%
\newcommand{\hlkwa}[1]{\textcolor[rgb]{0.161,0.373,0.58}{\textbf{#1}}}%
\newcommand{\hlkwb}[1]{\textcolor[rgb]{0.69,0.353,0.396}{#1}}%
\newcommand{\hlkwc}[1]{\textcolor[rgb]{0.333,0.667,0.333}{#1}}%
\newcommand{\hlkwd}[1]{\textcolor[rgb]{0.737,0.353,0.396}{\textbf{#1}}}%
\let\hlipl\hlkwb

\usepackage{framed}
\makeatletter
\newenvironment{kframe}{%
 \def\at@end@of@kframe{}%
 \ifinner\ifhmode%
  \def\at@end@of@kframe{\end{minipage}}%
  \begin{minipage}{\columnwidth}%
 \fi\fi%
 \def\FrameCommand##1{\hskip\@totalleftmargin \hskip-\fboxsep
 \colorbox{shadecolor}{##1}\hskip-\fboxsep
     % There is no \\@totalrightmargin, so:
     \hskip-\linewidth \hskip-\@totalleftmargin \hskip\columnwidth}%
 \MakeFramed {\advance\hsize-\width
   \@totalleftmargin\z@ \linewidth\hsize
   \@setminipage}}%
 {\par\unskip\endMakeFramed%
 \at@end@of@kframe}
\makeatother

\definecolor{shadecolor}{rgb}{.97, .97, .97}
\definecolor{messagecolor}{rgb}{0, 0, 0}
\definecolor{warningcolor}{rgb}{1, 0, 1}
\definecolor{errorcolor}{rgb}{1, 0, 0}
\newenvironment{knitrout}{}{} % an empty environment to be redefined in TeX

\usepackage{alltt}

\usepackage[OT4]{polski}
\usepackage[utf8]{inputenc}
\usepackage[T1]{fontenc}
\usepackage[top=2.5cm, bottom=2.5cm, left=2cm, right=2cm]{geometry}
\usepackage{graphicx}
\usepackage{float}
\usepackage[colorlinks=true, linkcolor=blue]{hyperref}
\usepackage{amsmath}
\usepackage{amssymb}



\title{Lista 1}
\author{Mikołaj Langner, Marcin Kostrzewa}
\date{31.3.2021}
\IfFileExists{upquote.sty}{\usepackage{upquote}}{}
\begin{document}
\maketitle

\section{Etap I}
Wczytajmy dane z pliku i przeprowadźmy ich wstępną analizę i obróbkę:
\begin{knitrout}
\definecolor{shadecolor}{rgb}{0.969, 0.969, 0.969}\color{fgcolor}\begin{kframe}
\begin{alltt}
\hlstd{df} \hlkwb{<-} \hlkwd{read.csv}\hlstd{(}\hlstr{'churn.txt'}\hlstd{,} \hlkwc{stringsAsFactors} \hlstd{=} \hlnum{TRUE}\hlstd{)}
\end{alltt}
\end{kframe}
\end{knitrout}

\begin{itemize}
\item sprawdźmy ich typ.
% latex table generated in R 4.0.4 by xtable 1.8-4 package
% Sun Mar 28 16:35:35 2021
\begin{table}[H]
\centering
\begin{tabular}{rl}
  \hline
 & Typ zmiennej \\ 
  \hline
State & factor \\ 
  Account.Length & integer \\ 
  Area.Code & integer \\ 
  Phone & factor \\ 
  Int.l.Plan & factor \\ 
  VMail.Plan & factor \\ 
  VMail.Message & integer \\ 
  Day.Mins & numeric \\ 
  Day.Calls & integer \\ 
  Day.Charge & numeric \\ 
  Eve.Mins & numeric \\ 
  Eve.Calls & integer \\ 
  Eve.Charge & numeric \\ 
  Night.Mins & numeric \\ 
  Night.Calls & integer \\ 
  Night.Charge & numeric \\ 
  Intl.Mins & numeric \\ 
  Intl.Calls & integer \\ 
  Intl.Charge & numeric \\ 
  CustServ.Calls & integer \\ 
  Churn. & factor \\ 
   \hline
\end{tabular}
\caption{Hello} 
\label{tab:tabela1}
\end{table}


\item sprawdźmy czy pojawiają się wartości brakujące:
\end{itemize}

\section{Etap II}


\section{Etap III}


\section{Etap IV}


\end{document}
